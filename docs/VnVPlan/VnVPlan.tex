\documentclass[12pt, titlepage]{article}

\usepackage{booktabs}
\usepackage{tabularx}
\usepackage{hyperref}
\hypersetup{
    colorlinks,
    citecolor=blue,
    filecolor=black,
    linkcolor=red,
    urlcolor=blue
}
\usepackage[round]{natbib}
\usepackage{amsmath}

\input{../Comments}
\input{../Common}


\begin{document}

\title{System Verification and Validation Plan for Re-ProtGNN} 
\author{\authname}
\date{\today}
	
\maketitle

\pagenumbering{roman}

\section*{Revision History}

\begin{tabularx}{\textwidth}{p{3cm}p{2cm}X}
\toprule {\bf Date} & {\bf Version} & {\bf Notes}\\
\midrule
Feb 24 & 1.0 & Initial Draft\\
April 10 & 2.0 & Updated by Feedback\\
April 12 & 3.0 & Updated by Feedback\\
April 13 & 4.0 & Updated by Feedback\\
April 18 & 5.0 & Final Version\\
\bottomrule
\end{tabularx}

~\\


\newpage

\tableofcontents

\listoftables

%section: list of figures
%\listoffigures

\newpage

\section{Symbols, Abbreviations, and Acronyms}

\renewcommand{\arraystretch}{1.2}
\begin{tabular}{l l} 
  \toprule		
  \textbf{symbol} & \textbf{description}\\
  \midrule 
  T & Test\\
  A & Assumption\\
  R & Requirement\\
  SRS & Software Requirements Specification\\
  ProtGNN & Prototype-based Graph Neural Network\\
  Re-ProtGNN & Re-implementation of the ProtGNN model\\
  GNN & Graph Neural Network\\
  GIN & Graph Isomorphism Network\\
  \bottomrule
\end{tabular}\\

\newpage

\pagenumbering{arabic}

This document outlines the Verification and Validation (VnV) plan for Re-ProtGNN, a prototype-based interpretable Graph Neural Network. The objective of this plan is to ensure that the system meets the functional and non-functional requirements specified in the Software Requirements Specification (SRS). The document is structured as follows: Section 2 provides general information about Re-ProtGNN, including its objectives and scope. Section 3 details the verification strategies, covering SRS verification, design verification, and implementation verification. Section 4 describes the system tests, including functional and non-functional testing. Finally, Section 5 will cover additional test descriptions as needed.

\section{General Information}

\subsection{Summary}

The software under validation is Re-ProtGNN, a model designed to enhance the interpretability of Graph Neural Networks. It aims to classify graph-structured data while generating prototypes that explain its predictions. The system consists of two main components:

\begin{itemize}
    \item Training Phase: Learns meaningful representations of graphs while optimizing a loss function to improve both classification accuracy and prototype relevance.
    \item Inference Phase: Uses the trained model to classify unseen graphs and generate a set of representative prototypes that provide human-interpretable explanations.
\end{itemize}

Re-ProtGNN is implemented in Python, utilizing PyTorch Geometric for graph learning and Pytest for automated testing.


\subsection{Objectives}

The main goal of this VnV plan is to verify the correctness of the program, which includes ensuring that the system correctly processes graph-structured inputs, trains models effectively, and generates prototype-based explanations.

%\begin{itemize}
    %\item Correctness: Ensure that the system correctly processes graph-structured inputs, trains models effectively, and generates meaningful prototype-based explanations.
    %\item Accuracy: Assess whether the model attains a minimum classification accuracy of 80\% on the MUTAG dataset and confirm that the selected prototypes align with expected graph representations.
    %item Interpretability: Validate that the extracted prototypes contribute to a clearer understanding of the model’s decision-making process.
%\end{itemize}

Out-of-Scope Objectives:
\begin{itemize}
    \item External Library Verification: Core dependencies such as PyTorch and Torch-Geometric are presumed to be reliable and are not explicitly tested in this plan.
    \item Graph Encoder Validation: Graph encoders, including Graph Isomorphism Networks (GINs), are adopted from prior research, and their correctness is assumed without additional validation.
\end{itemize}

\subsection{Challenge Level and Extras}
This is a research project, and no additional components will be included due to time constraints.

\subsection{Relevant Documentation}

The Re-ProtGNN project is supported by several key documents that ensure the system is properly designed, implemented, and validated. These documents include:

\begin{itemize}
    \item Problem Statement: This document~\citep{yuanqi2025protgnn} introduces the motivation of Re-ProtGNN and the core problem it aims to solve.
    
    \item Software Requirements Specification (SRS): The SRS~\citep{Yuanqi_ReProtGNN_SRS} defines the functional and non-functional requirements of Re-ProtGNN, and it also outlines the expected behavior of the system.
    
    \item Verification and Validation (VnV) Plan: The VnV Plan~\citep{Yuanqi_ReProtGNN_VnV} describes the testing strategy used to verify the system’s functionality and evaluate its correctness.
    
    \item Module Guide (MG): The MG~\citep{Yuanqi_ReProtGNN_MG} outlines the design decisions, responsibilities, and internal details of each module to support understanding and modular testing.
    
    \item Module Interface Specification (MIS): The MIS~\citep{Yuanqi_ReProtGNN_MIS} specifies the interface, access routines, and expected behavior of each module, enabling consistent implementation and verification.
\end{itemize}


\section{Plan}

This section outlines the Verification and Validation (VnV) plan for Re-ProtGNN. It starts with an introduction to the verification and validation team (Subsection~\ref{sec:vvt}), detailing the members and their roles. Next, it covers the SRS verification plan (Subsection~\ref{sec:srsvp}), followed by the design verification plan (Subsection ~\ref{sec:dvp}). The document then presents the VnV verification plan (Subsection~\ref{sec:vvp}) and the implementation verification plan (Subsection~\ref{sec:ivp}). Finally, it includes automated testing and verification tools (Subsection~\ref{sec:att}) and concludes with the software validation plan (Subsection~\ref{sec:svp}).

\subsection{Verification and Validation Team}
\label{sec:vvt}

\begin{table}[h]
    \centering
    \begin{tabular}{|l|l|l|p{5cm}|}
        \hline
        \textbf{Name} & \textbf{Document} & \textbf{Role} & \textbf{Description} \\ 
        \hline
        Yuanqi Xue & All & Author & Create and manage all required documents, develop the VnV plan, conduct VnV testing, and verify the implementation. \\ 
        \hline
        Dr. Spencer Smith & All & Instructor/ Reviewer & Review all the documents. \\ 
        \hline
        Yinying Huo & All & Domain Expert & Review all the documents. \\ 
        \hline
    \end{tabular}
    \caption{Verification and Validation Team}
\end{table}

\subsection{SRS Verification Plan}
\label{sec:srsvp}
An initial review of the SRS will be conducted by Dr. Spencer Smith and Yinying Huo to ensure its accuracy, completeness, and feasibility. The review will follow a manual inspection process using an SRS Checklist~\citep{Yuanqi_ReProtGNN_SRSChecklist} to assess the clarity and alignment of SRS with project objectives.

Reviewers will provide feedback via GitHub issues, and Yuanqi Xue, as the author, will be responsible for addressing revisions.

\subsection{Design Verification Plan}
\label{sec:dvp}
The design verification process, covering the Module Guide (MG) and Module Interface Specification (MIS), will be conducted by Dr. Spencer Smith and domain expert Yinying Huo. Their feedback will be shared through GitHub issues, and Yuanqi Xue will be responsible for making the necessary revisions.

To ensure a structured verification process, Dr. Spencer Smith has created an MG checklist~\citep{Yuanqi_ReProtGNN_MGChecklist} and MIS checklist~\cite{Yuanqi_ReProtGNN_MGChecklist}, both of which will be used to evaluate clarity, consistency, and correctness in the design documents. The verification process will ensure that the system architecture, module interactions, and design choices align with the project objectives.

\subsection{Verification and Validation Plan Verification Plan}
\label{sec:vvp}
The Verification and Validation (VnV) Plan will be reviewed by Dr. Spencer Smith and domain expert Yinying Huo to ensure that the validation methodology aligns with the project’s objectives. Feedback from reviewers will be provided via GitHub issues, where all necessary revisions and updates will be documented.

To maintain consistency and thorough evaluation, a VnV checklist~\citep{Yuanqi_ReProtGNN_VnVChecklist} prepared by Dr. Spencer Smith will be used to assess the completeness, correctness, and applicability of the verification and validation process. 


\subsection{Implementation Verification Plan}
\label{sec:ivp}

The implementation of Re-ProtGNN will be verified through a combination of unit tests and system tests. This verification process ensures that both functional and non-functional requirements are satisfied, as detailed in Section~\ref{sec:system-tests}. Unit tests will target core components of the system, such as data preprocessing, model inference, and prototype generation, to confirm correctness at the module level. System tests will evaluate the end-to-end performance of Re-ProtGNN by validating input data, confirming training convergence, and verifying that the correct number of prototypes is generated. The specific testing tools and methodologies employed are described in Section~\ref{subsec: verification-tools}, and the corresponding test cases are presented in Section~\ref{sec:system-tests}.





\subsection{Automated Testing and Verification Tools}
\label{sec:att}
\label{subsec: verification-tools}
Re-ProtGNN will use a combination of automated testing and verification tools to ensure code correctness:

Unit Testing:
Pytest will be used for testing individual components, including data processing, loss functions, and inference logic. These tests will help verify that each module functions as expected before integration into the full system.

Continuous Integration (CI):
GitHub Actions will be configured to automate testing after each commit. The workflow will include:
\begin{itemize}
    \item Running unit tests to verify functionality.
    \item Performing data integrity checks.
    \item Validating dependencies to prevent compatibility issues.
\end{itemize}



\subsection{Software Validation Plan}
\label{sec:svp}
A testing dataset (i.e., a separate 20\% test split from the MUTAG dataset~\citep{debnath1991structure}) will be used to assess the model’s classification effectiveness and the clarity of its prototype-based explanations, with accuracy serving as the main benchmark.

\section{System Tests}
\label{sec:system-tests}
This section outlines the system tests designed to evaluate both functional and non-functional requirements.

\subsection{Tests for Functional Requirements}
\label{sub:FR}
This section outlines the tests designed to validate the functional requirements of Re-ProtGNN, ensuring the system behaves as expected under various conditions. The testing areas include input verification, model training correctness, and inference validation, which corresponds to the R1, R2, and R3 of \href{https://github.com/Yuanqi-X/Re-ProtGNN/blob/main/docs/SRS/SRS.pdf}{SRS} ~\citep{Yuanqi_ReProtGNN_SRS}.


\subsubsection{Area of Testing 1: Input Validation}

This section ensures the system can correctly detect and handle invalid or missing input files before dataset processing, as required to fulfill input robustness and error traceability.

\paragraph{Test for Raw Input Integrity}

\begin{enumerate}

\item{T1: Missing File Handling Test\\}

Control: Automatic

Initial State: Dataset folder exists, but no raw input files are present.

Input: A folder named MUTAG/raw with no content.

Output: The system should raise a FileNotFoundError indicating which required raw file is missing.

Test Case Derivation: Ensures that the system validates file existence prior to processing. This corresponds to preconditions for requirement R1.

How test will be performed: A Pytest script sys\_test\_input\_validation.py creates an empty raw input directory and initializes the MUTAG dataset wrapper. It passes if FileNotFoundError is raised and the error message references the correct file path.

\vspace{0.3em}

\item{T2: Format Consistency and Type Validation Test\\}

Control: Automatic

Initial State: All required files are present but contain incorrectly formatted (non-numeric or corrupted) content.

Input: A folder named MUTAG/raw with files such as MUTAG\_A.txt and MUTAG\_graph\_labels.txt containing malformed strings.

Output: The system should raise a ValueError if the file content is not parseable (e.g., containing strings instead of numerical values).

Test Case Derivation: Ensures that the dataset loader enforces structural correctness, file consistency, and semantic type assumptions.

How test will be performed: A Pytest script sys\_test\_input\_validation.py populates all required files with junk data (e.g., strings instead of integers) and verifies that a ValueError is raised during parsing or validation.

\vspace{0.3em}

\item{T3: Valid Dataset Load Test\\}

Control: Automatic

Initial State: The full MUTAG dataset is properly placed under the ./data/MUTAG directory with correctly formatted raw files.

Input: All required raw text files (adjacency, node labels, graph labels, and graph indicators) with valid data.

Output: The dataset should load successfully, returning a valid dataset instance with length greater than zero. No exceptions should be raised.

Test Case Derivation: Confirms the system can successfully load and parse a standard benchmark dataset, satisfying R1 under nominal conditions.

How test will be performed: A Pytest script sys\_test\_input\_validation.py initializes the dataset wrapper using the default data path and asserts the dataset loads correctly and contains nonzero graphs.

\end{enumerate}


\subsubsection{Area of Testing 2: Training Convergence}
\label{sub:convergence}
This test verifies that the model training process optimizes the objective function over time. It ensures that the model effectively minimizes training loss, thereby satisfying the requirement for stable and meaningful training behavior.

\paragraph{Test for Training Loss Convergence}

\begin{enumerate}

\item{T2: Loss Convergence Test\\}

Control: Automatic

Initial State: The training pipeline is configured using the default parameters, and the training dataset is loaded from the dataset folder. No manual intervention is required.

Input: A graph classification dataset split into training, validation, and test sets; an untrained model initialized with random weights; and the configuration parameters defined in the system (e.g., maximum number of epochs, early stopping patience).

Output: A sequence of training loss values recorded after each epoch. The final loss should be less than or equal to the initial loss, indicating convergence.

Test Case Derivation: This test confirms that the model satisfies the training requirement by reducing the loss function across epochs, as expected in successful optimization procedures. If the training loss increases over time or fluctuates abnormally, the test will fail, revealing instability in training.

How the test will be performed:
\begin{itemize}
    \item In the Pytest script sys\_test\_loss\_converge.py, The system’s main training pipeline is launched using the main entry point with loss coefficient values set to zero to isolate base training behavior.
    \item The append\_record function is temporarily patched to intercept and store loss values across epochs in a list.
    \item Once training is complete, the list of recorded losses is analyzed.
    \item The test asserts the condition that the final loss must be less than or equal to the initial loss, indicating that training successfully reduced the loss.
\end{itemize}

A successful result can demonstrates that the system is capable of learning meaningful representations in the training phase.

\end{enumerate}



\subsubsection{Area of Testing 3: Inference Verification}
This test ensures that the trained model can correctly classify unseen test samples and that the predictions returned during the inference phase are complete, numerically valid, and structurally consistent.

\paragraph{Test for Inference Accuracy}

\begin{enumerate}

\item{T3: Inference Accuracy Test\\}

Control: Automatic

Initial State: A trained model checkpoint is saved in the src/checkpoint directory and available for loading.

Input: A test data loader produced from the configured dataset split, and a graph neural network model initialized and updated using the saved checkpoint weights.

Output: The system should return a scalar classification accuracy value, a scalar loss value, a 1D tensor of predicted class labels, and a 2D tensor or array of class probability scores for each test instance.

Test Case Derivation: This test verifies that the trained model satisfies R3 in SRS~\citep{Yuanqi_ReProtGNN_SRS} by making predictions on the test set and reporting classification performance. It also confirms that the returned predictions and probabilities are well-formed and aligned with the dataset.

How test will be performed: 
\begin{itemize}
    \item In the Pytest script sys\_test\_input\_validation.py, the test begins by loading the full dataset and applying a fixed ratio-based split into training, validation, and test sets using load\_dataset().
    \item The model is constructed using the input and output dimensions inferred from the dataset and updated with the best checkpoint saved during training.
    \item Inference is performed on the test split using run\_inference().
    \begin{itemize}
        \item The returned dictionary contains the keys acc (accuracy) and loss, both of which must be real-valued scalars.
        \item The accuracy must be strictly greater than 50\%, which confirms that the model is performing above chance level on a binary classification task.
        \item The shape of the prediction tensor all\_preds must match the number of samples in the test set, meaning that there is exactly one predicted label for every input graph.
        \item The probability tensor all\_probs must be two-dimensional, with shape \((N, C)\), where \(N\) is the number of test samples and \(C\) is the number of classes. This ensures that the model outputs a complete probability distribution for each test instance.
        \item The first dimension (number of rows) of all\_probs must exactly equal the length of all\_preds, confirming that the predicted label and associated probability scores correspond to the same set of inputs.
    \end{itemize}
\end{itemize}

The classification accuracy is computed using the formula:å
\[
\text{Accuracy} = \frac{\text{Number of Correct Predictions}}{\text{Total Number of Data Points}} = \frac{1}{N} \sum_{i=1}^{N} 1(\hat{y}_i = y_i)
\]
where \(N\) is the total number of graphs in the test set, \(\hat{y}_i\) is the predicted label for the \(i\)th graph, and \(y_i\) is the corresponding ground-truth label. The indicator function \(1(\cdot)\) evaluates to 1 if the predicted label is correct, and 0 otherwise.

\end{enumerate}




\subsection{Tests for Nonfunctional Requirements}

This section outlines the tests designed to verify the nonfunctional requirements of Re-ProtGNN, including accuracy, usability, and portability.

\subsubsection{Reliability}
		
The reliability of the software is demonstrated through the execution of functional requirement tests, as described in Section~\ref{sub:FR} and Section~\ref{sub:unit FR}. These tests ensure the system behaves as expected under defined conditions.


%\begin{enumerate}

%\item{T4: Accuracy Test\\}

%Type: Automated
					
%Initial State: A trained Re-ProtGNN model is loaded.
					
%Input: The testing dataset.
					
%Output: The system should achieve a classification accuracy of at least 80\%.
					
%How test will be performed: A Pytest script will run inference on the test dataset and compare the predicted labels with ground truth. The script will calculate the overall classification accuracy and ensure it meets the required threshold ($\geq 80\%$). The accuracy percentage will be logged to evaluate model performance.

%\end{enumerate}

\subsubsection{Usability}

The usability of the software is measured mainly by the Usability Survey in Section~\ref{sub:survey}.

\begin{enumerate}
  \item{T4: Usability Survey Test\\}
  Type: Manual \\
  Initial State: The software is fully set up and operational \\
  Input/Condition: None \\
  Output/Result: Completed survey responses from the user \\
  Test Case Derivation: Measures user perception of ease of use and clarity of the interface, as defined by usability metrics. \\
  How test will be performed: The user will be asked to complete a survey after using the system. The survey questions are listed in Appendix~\ref{sub:survey}.
\end{enumerate}

%\begin{enumerate}
%    \item{T5: Usability Test\\}

%    Type: Manual
    					
%    Initial State: Trained Re-ProtGNN model is available.
    					
%    Input: A set of graphs with learned prototypes.
    					
%    Output: The system should visualize prototypes as graph images, making them more interpretable than raw adjacency matrices or node feature representations.
    					
%    How test will be performed: A set of sample graphs will be processed through the system, and the generated prototype visualizations will be reviewed for clarity and interpretability. Testers will manually verify whether the images effectively represent key graph structures and provide meaningful insights into model decisions. Feedback will be collected to assess usability improvements.
%\end{enumerate}

%\subsubsection{Area of Testing3: Portability}

%\paragraph{Portability}

%\begin{enumerate}
%    \item{T6: Portability Test\\}

%    Type: Manual
    					
%    Initial State: None
    					
%    Condition: The users should have PyTorch 1.8.0 and Torch-Geometric 2.0.2 installed.
    					
%    Result: The system should successfully run on users' machines, with all functions working as expected.
    					
%    How test will be performed: The author, Yuanqi Xue, will install the software on Windows, macOS, and Linux systems, verifying that it runs correctly across all platforms.
%\end{enumerate}

\subsection{Traceability Between Test Cases and Requirements}

\begin{table}[h!]
  \centering
  \begin{tabular}{|c|c|c|c|c|}
  \hline
    & T1 & T2 & T3 & T4 \\
  \hline
  R1        & X  &   &   &    \\ \hline
  R2        &   & X  &   &   \\ \hline
  R3        &   &   & X  &    \\ \hline
  NFR1      & X  & X  & X  &    \\ \hline
  NFR2      &   &   &   & X   \\ \hline
  \end{tabular}
  \caption{Traceability Matrix of Test Cases and Requirements}
  \label{Table:trace-test-req}
  \end{table}


\section{Unit Test Description}

\subsection{Unit Testing Scope}

This section outlines the unit testing coverage across the module hierarchy presented in Section~5. Unit tests were implemented for all leaf modules that contain self-contained functionality. Specifically, we provide full coverage for:

\begin{itemize}
  \item \textbf{M3 - Input Format Module:} This module includes dataset loading, preprocessing, and dataloader construction. Unit tests validate both correct data splits and failure cases such as missing or malformed input files.

  \item \textbf{M5 - Training Module:} This module governs the model training procedure, including loss computation, prototype projection, mode configuration, and evaluation. Unit tests exercise all internal access routines except for the top-level train() function, which is validated through full-system tests (e.g., loss convergence test in~\ref{sub:convergence}).
  
  \item \textbf{M6 - Output Visualization Module:} This module handles logging, checkpoint saving, and subgraph visualization. Unit tests verify file-writing routines, model serialization behavior, and visual explanation dispatching logic for supported datasets.
  
  \item \textbf{M8 - Inference Module:} This module runs model evaluation on test data and reports metrics. Unit tests confirm correct loss/accuracy aggregation, prediction concatenation, and logging calls.
  
  \item \textbf{M9 - Explanation Module:} This module implements the subgraph explanation strategy. Unit tests cover internal score computation, prototype similarity matching, recursive backpropagation, and high-level explanation behavior.
\end{itemize}

\paragraph{Rationale for Non-Tested Modules.}
The following modules are not unit-tested either due to their indirect role, external origin, or verification through integration/system-level tests:

\begin{itemize}
  \item M1 - Hardware-Hiding Module: This module abstracts disk I/O and memory operations. Its functionality is implicitly tested through file saving/loading operations in other modules, such as checkpoint storage and dataset caching.
  
  \item M2 - Configuration Module: This module only contains global constants and command-line configuration structures. Since it defines no active computation, it is not unit-tested but is covered implicitly through the modules that consume its parameters.
  
  \item M4 - Control Module: This module is the main program and serves as the orchestrator for all other modules. Its correctness is indirectly verified through the unit tests of the functional modules it calls.
  
  \item M7 - Model Module: This module imports standard GNN backbones from published codebases. These components are assumed correct and excluded from internal unit testing.
  \item M10, M11, M12 - Pytorch Module, Pytorch Geometric Module, GUI Module: These modules are external libraries (i.e., Pytorch, Pytroch Geometric, and Matplotlib) and are assumed correct.
\end{itemize}




\subsection{Tests for Functional Requirements}
\label{sub:unit FR}

\subsubsection{Input Format Module (M3)}

This module corresponds to the data ingestion and preprocessing layer responsible for loading datasets, applying transformations, and preparing batched dataloaders for graph classification. The tests below follow the specifications outlined in the MIS, covering access programs such as load\_dataset, \_get\_dataloader, and include wrapper functions for MUTAG and other supported datasets. Each test is derived from either a normal usage pattern or a defined edge case. The goal is to verify expected input-output behavior and robust error handling.

\begin{enumerate}

\item{test-M3-1: Dataset Loading Interface Test\\}
Type: Automatic, Functional \\
Initial State: No dataset is currently loaded \\
Input: Dataset name (e.g., MUTAG), dataset path, and training configuration \\
Output: Dataset object with valid num\_node\_features and num\_classes; a dictionary containing train, eval, and test DataLoaders \\
Test Case Derivation: Verifies the behavior of load\_dataset() and confirms correct metadata extraction and data loader construction. \\
How test will be performed: Patch internal helpers (\_get\_dataset, \_get\_dataloader), call load\_dataset(), and assert on returned structure and contents.

\item{test-M3-2: Random Split Dataloader Construction Test\\}
Type: Automatic, Functional \\
Initial State: A PyTorch Geometric dataset object is available \\
Input: List of synthetic Data objects, split ratio [0.6, 0.2, 0.2] \\
Output: DataLoader dictionary with train, eval, and test splits totaling original size \\
Test Case Derivation: Ensures that randomized dataset splitting logic in \_get\_dataloader() produces reproducible and valid splits \\
How test will be performed: Pass a dataset with batch size and split ratio into \_get\_dataloader() and  assert that the returned dictionary contains keys ``train'', ``eval'', and ``test''. 


\item{test-M3-3: Predefined Split Index Test\\}
Type: Automatic, Functional \\
Initial State: Dataset includes a supplement['split\_indices'] attribute \\
Input: Dataset with split indices tensor defining train, eval, and test members \\
Output: Three DataLoaders corresponding to these index masks \\
Test Case Derivation: Required for compatibility with datasets that ship with predefined splits. \\
How test will be performed: Mock dataset with index labels and verify correct construction of DataLoaders using subset masks.

\item{test-M3-4: Dataset Wrapper Invocation Test (MUTAG)\\}
Type: Automatic, Functional \\
Initial State: No dataset loaded \\
Input: Dataset name MUTAG and directory path \\
Output: \_MUTAGDataset object loaded from saved data.pt \\
Test Case Derivation: Verifies correct dispatching from \_get\_dataset() to MUTAG-specific wrapper and confirms the dataset inherits InMemoryDataset \\
How test will be performed: Patch torch.load and process(), create \_MUTAGDataset, and validate class and internal attributes.

\item{test-M3-5: Unsupported Dataset Handling Test\\}
Type: Automatic, Functional \\
Initial State: No dataset selected \\
Input: Dataset name not recognized by dispatch (e.g., `FakeDataset') \\
Output: NotImplementedError is raised \\
Test Case Derivation: Confirms graceful failure when an invalid or unsupported dataset name is passed \\
How test will be performed: Pass a fake name into \_get\_dataset() and check the correct exception is raised.


\end{enumerate}


\subsubsection{Training Module (M5)}

The Training Module coordinates the warm-up, projection, and full training phases of the prototype-based GNN. It computes loss, handles optimizer updates, evaluates validation performance, manages prototype alignment, and triggers checkpoint saving. Unit tests are implemented to cover all internal access routines as specified in the MIS, and to isolate individual components of the training logic such as loss computation, dataset statistics, training mode configuration, and prototype behavior. The primary training loop itself is indirectly tested through system-level loss convergence (see~\ref{sub:convergence}).

\begin{enumerate}

\item{test-M5-1: Total Loss Computation Test\\}
Type: Automatic, Functional \\
Initial State: Dummy model with initialized prototype identity and parameters \\
Input: Logits, class labels, minimum distances, loss function, and regularization weights \\
Output: Scalar tensor representing the total loss \\
Test Case Derivation: Ensures correct aggregation of classification, cluster, separation, L1, and diversity losses. \\
How test will be performed: Generate synthetic logits and distances, and pass them into \_compute\_total\_loss().

\item{test-M5-2: Dataset Statistics Logging Test\\}
Type: Automatic, Functional \\
Initial State: Dataset of graphs with fixed node and edge counts \\
Input: Dataset list of PyG-compatible data objects \\
Output: Printed statistics and log string \\
Test Case Derivation: Logs dataset-level statistics for monitoring purposes. \\
How test will be performed: Use capsys to capture standard output of \_log\_dataset\_stats() and verify its format.

\item{test-M5-3: Training Mode Configuration Test\\}
Type: Automatic, Functional \\
Initial State: Model with trainable GNN layers and classifier head \\
Input: warm\_only=True and False flags \\
Output: Parameters’ requires\_grad state adjusted accordingly \\
Test Case Derivation: Distinguishes between warm-up and full joint training modes. \\
How test will be performed: Call \_set\_training\_mode() and ensure no exceptions and state changes occur.

\item{test-M5-4: Prototype Projection Test\\}
Type: Automatic, Functional \\
Initial State: Model and dataset with one-hot labels \\
Input: Dataset, prototype vectors, and prototype-to-label mapping \\
Output: Projected prototype vector aligned to closest explanation \\
Test Case Derivation: Validates the projection of each prototype to the most similar real sample. \\
How test will be performed: Patch get\_explanation() and call \_project\_prototypes() to check for correct assignment and execution.

\item{test-M5-5: Evaluation Routine Test\\}
Type: Automatic, Functional \\
Initial State: Model in evaluation mode with known output predictions \\
Input: Validation dataloader and loss function \\
Output: Mean loss and accuracy across batches \\
Test Case Derivation: Measures performance statistics for early stopping and model selection. \\
How test will be performed: Use a model with fixed logits and assert expected output keys in the result dictionary.

\end{enumerate}

\paragraph{Additional Note:}
While the top-level train() function is not directly unit-tested, it is indirectly verified through system test T2 (see~\ref{sub:convergence}), which performs end-to-end training runs and confirms expected loss convergence over epochs. This test implicitly validates the integration of training components such as optimizer updates, prototype projection, and checkpoint logic.




\subsubsection{Output Visualization Module (M6)}

This module handles logging, model checkpoint saving, and visual explanation rendering for subgraph-based GNN interpretability. Unit tests have been designed to cover each public access routine defined in the MIS for this module, including standard operations for logging, model saving, and drawing subgraphs across supported datasets. The tests address both expected behavior and edge case handling (e.g., unsupported dataset types). The module’s correctness is validated by asserting the file writes and method calls.

\begin{enumerate}

\item{test-M6-1: Logging Append Test\\}
Type: Automatic, Functional \\
Initial State: Log file is empty or preexisting \\
Input: A single-line string (e.g., test log line) \\
Output: The string is appended to the ./results/log/hyper\_search log file with a newline character \\
Test Case Derivation: Validates the basic logging utility used across training and inference scripts. \\
How test will be performed: Patch open() and check that the file was opened in append mode and write() was called with the correct formatted input.

\item{test-M6-2: Model Save with Best Checkpoint\\}
Type: Automatic, Functional \\
Initial State: A model object is instantiated \\
Input: Checkpoint directory, epoch number, model, model name, accuracy value, and is\_best=True flag \\
Output: A checkpoint file is saved, and a copy is made to the best checkpoint path \\
Test Case Derivation: Ensures that the best model is correctly tracked and persisted \\
How test will be performed: Patch torch.save and shutil.copy, then assert both were called correctly.

\item{test-M6-3: Model Save without Best Checkpoint\\}
Type: Automatic, Functional \\
Initial State: A model object is instantiated \\
Input: Checkpoint directory, epoch number, model, model name, accuracy value, and is\_best=False flag \\
Output: A checkpoint file is saved, but no copy is made for best checkpoint \\
Test Case Derivation: Validates that saving logic bypasses best checkpoint logic when not applicable \\
How test will be performed: Patch torch.save and shutil.copy, and assert only the save call is invoked.

\item{test-M6-4: Explanation Dispatcher Test (MUTAG)\\}
Type: Automatic, Functional \\
Initial State: ExpPlot initialized with dataset name \\
Input: A NetworkX graph, list of nodes to highlight, and a feature matrix \\
Output: Call to \_draw\_molecule() subroutine with appropriate arguments \\
Test Case Derivation: Ensures correct routing for molecule-based explanations \\
How test will be performed: Patch \_draw\_molecule method and assert it is called with expected input.

\item{test-M6-5: Subgraph Drawing Output Test\\}
Type: Automatic, Functional \\
Initial State: ExpPlot object instantiated with a supported dataset name \\
Input: A NetworkX graph and temporary output file path \\
Output: An image file (e.g., PNG) is saved containing the visualized subgraph \\
Test Case Derivation: Basic requirement that the visualization function executes without error \\
How test will be performed: Create a test graph, call \_draw\_subgraph(), and verify that the file exists at the output location.

\item{test-M6-6: Dataset Routing Failure Test\\}
Type: Automatic, Functional \\
Initial State: ExpPlot initialized with an unsupported dataset name \\
Input: Any graph and a list of nodes \\
Output: A NotImplementedError is raised \\
Test Case Derivation: Prevents silent failures for unrecognized datasets in the drawing logic \\
How test will be performed: Patch matplotlib.pyplot.savefig to suppress side effects and assert that an error is raised when drawing with an invalid dataset name.

\end{enumerate}




\subsubsection{Inference Module (M8)}

This module encapsulates the inference-time behavior of the trained GNN model. It is responsible for computing predictions, evaluating performance on unseen data, and logging results. The primary access routine run\_inference is verified using unit tests that validate its output structure, accuracy computation, loss aggregation, and logging side effects. Both functional and boundary conditions (e.g., single-batch datasets) are tested to ensure stability across usage scenarios defined in the MIS.

\begin{enumerate}

\item{test-M8-1: Inference Execution and Output Test\\}
Type: Automatic, Functional \\
Initial State: Trained model and loss function loaded \\
Input: Evaluation dataloader, model object, and loss function \\
Output: Dictionary with loss and accuracy, prediction logits and probabilities \\
Test Case Derivation: Ensures the function executes end-to-end and returns all expected fields. \\
How test will be performed: Pass a dataloader and model to the inference routine and verify return structure.

\item{test-M8-2: Accuracy Calculation Verification\\}
Type: Automatic, Functional \\
Initial State: Model returns known logits and true labels \\
Input: Model with fixed output and corresponding dataloader \\
Output: Correct accuracy calculation in result dictionary \\
Test Case Derivation: Verifies the correctness of the accuracy logic based on class match count. \\
How test will be performed: Use controlled outputs from a model to assert expected accuracy values.

\item{test-M8-3: Inference Logging Test\\}
Type: Automatic, Functional \\
Initial State: Logging file initialized or exists \\
Input: Same as above \\
Output: Record appended to log file \\
Test Case Derivation: The system is expected to log inference results consistently. \\
How test will be performed: Patch the logging function and verify it is invoked with correctly formatted result.

\item{test-M8-4: Batch Aggregation Consistency Test\\}
Type: Automatic, Functional \\
Initial State: Model returns outputs in multiple batches \\
Input: Multi-batch dataloader and inference model \\
Output: Concatenated arrays for predictions and probabilities \\
Test Case Derivation: All batches should be merged for final output consistency. \\
How test will be performed: Provide two or more batches and check that final arrays match the total batch size.

\end{enumerate}


\subsubsection{Explanation Module (M9)}

The Explanation Module is responsible for identifying and visualizing the most relevant subgraph for a given GNN prediction using a Monte Carlo Tree Search (MCTS)–based strategy. The associated unit tests ensure correctness of explanation scores, rollout behavior, subgraph selection, and visualization triggering. Tests are selected to exercise both the black-box interface and key white-box mechanisms described in the MIS, such as prototype similarity scoring, node expansion, and recursive score propagation.

\begin{enumerate}

\item{test-M9-1: MCTS Node Scoring Test\\}
Type: Automatic, Functional \\
Initial State: Node created with initialized state parameters \\
Input: A coalition and dummy graph \\
Output: Valid computed $Q$ and $U$ scores \\
Test Case Derivation: Verifies the scoring behavior of tree nodes used in exploration phase \\
How test will be performed: Instantiate a node and validate returned $Q$ and $U$ values

\item{test-M9-2: Prototype Similarity Test\\}
Type: Automatic, Functional \\
Initial State: Model in evaluation mode with fixed embedding output \\
Input: Coalition and prototype tensor \\
Output: A scalar similarity score \\
Test Case Derivation: Ensures prototype-sample similarity is correctly measured \\
How test will be performed: Use a dummy GNN to compute the similarity for a given coalition

\item{test-M9-3: MCTS Scoring Utility Test\\}
Type: Automatic, Functional \\
Initial State: Multiple candidate nodes with varying priors \\
Input: List of MCTS nodes and scoring function \\
Output: List of computed scores \\
Test Case Derivation: Verifies score selection logic in exploration step \\
How test will be performed: Run scoring on multiple nodes and check ordering and values

\item{test-M9-4: MCTS Rollout Test\\}
Type: Automatic, Functional \\
Initial State: Root node initialized with full coalition \\
Input: Graph and scoring function \\
Output: Updated node visit counts and scores \\
Test Case Derivation: Validates recursive search and backpropagation logic \\
How test will be performed: Perform a rollout and assert structural updates

\item{test-M9-5: Explanation Function Test\\}
Type: Automatic, Functional \\
Initial State: Dummy GNN and graph sample \\
Input: Graph data, model, and prototype \\
Output: Most relevant subgraph node list, score, and embedding \\
Test Case Derivation: Checks output consistency for the high-level explanation function \\
How test will be performed: Call explanation API and verify output types and values


\item{test-M9-6: MCTS Rollout Backpropagation Test\\}
Type: Automatic, Functional \\
Initial State: Root with one child node with initialized score \\
Input: Score function \\
Output: Updated statistics at root node \\
Test Case Derivation: Verifies that child scores correctly propagate up the tree \\
How test will be performed: Trigger rollout and validate state updates on the parent
\end{enumerate}


\subsection{Tests for Nonfunctional Requirements}
Unit testing of nonfunctional requirements is considered out of scope for this project. These requirements will primarily be evaluated through the usability survey in Section~\ref{sub:survey}.

\subsection{Traceability Between Test Cases and Modules}
\label{sec:trace-test-modules}

This section provides evidence that all modules from the Module Hierarchy (Section~5) have been evaluated through targeted test cases where applicable. Each test case corresponds to a module or group of access routines within the module's responsibility.

\begin{table}[h!]
\centering
\begin{tabular}{|c|c|c|c|c|c|c|c|c|c|c|c|c|}
\hline
& M1 & M2 & M3 & M4 & M5 & M6 & M7 & M8 & M9 & M10 & M11 & M12 \\
\hline
test-M3-1   &     & X   & X   &  X   &     &     &     &     &     & X   & X   &     \\ \hline
test-M3-2   &     & X   & X   &  X   &     &     &     &     &     & X   & X   &     \\ \hline
test-M3-3   &     & X   & X   &  X   &     &     &     &     &     & X   & X   &     \\ \hline
test-M3-4   &     & X   & X   &  X   &     &     &     &     &     & X   & X   &     \\ \hline
test-M3-5   &     & X   & X   &  X   &     &     &     &     &     & X   & X   &     \\ \hline
test-M5-1   &     & X   &     &  X   & X   &     &     &     & X   & X   &     &     \\ \hline
test-M5-2   &     & X   &     &  X   & X   &     &     &     &     &     &     &     \\ \hline
test-M5-3   &     & X   &     &  X   & X   &     &     &     &     &     &     &     \\ \hline
test-M5-4   &     & X   &     &  X   & X   &     &     &     & X   & X   &     &     \\ \hline
test-M5-5   &     & X   &     &  X   & X   &     &     &     &     &     &     &     \\ \hline
test-M6-1   & X   &     &     &  X   &     & X   &     &     &     & X   &     & X   \\ \hline
test-M6-2   & X   &     &     &  X   &     & X   &     &     &     & X   &     & X   \\ \hline
test-M6-3   & X   &     &     &  X   &     & X   &     &     &     & X   &     & X   \\ \hline
test-M6-4   &     &     &     &  X   &     & X   &     &     &     &     &     &     \\ \hline
test-M6-5   &     &     &     &  X   &     & X   &     &     &     &     &     &     \\ \hline
test-M6-6   &     &     &     &  X   &     & X   &     &     &     &     &     &     \\ \hline
test-M8-1   &     &     &     &  X   &     &     & X   & X   &     & X   &     &     \\ \hline
test-M8-2   &     &     &     &  X   &     &     & X   & X   &     & X   &     &     \\ \hline
test-M8-3   &     &     &     &  X   &     &     & X   & X   &     & X   &     &     \\ \hline
test-M8-4   &     &     &     &  X   &     &     & X   & X   &     & X   &     &     \\ \hline
test-M9-1   &     & X   &     &  X   &     &     &     &     & X   & X   &     &     \\ \hline
test-M9-2   &     & X   &     &  X   &     &     &     &     & X   & X   &     &     \\ \hline
test-M9-3   &     & X   &     &  X   &     &     &     &     & X   & X   &     &     \\ \hline
test-M9-4   &     & X   &     &  X   &     &     &     &     & X   & X   &     &     \\ \hline
test-M9-5   &     & X   &     &  X   &     &     &     &     & X   & X   &     &     \\ \hline
test-M9-6   &     & X   &     &  X   &     &     &     &     & X   & X   &     &     \\ \hline
\end{tabular}
\caption{Traceability Matrix of Test Cases and Modules}
\label{Table:trace-test-modules}
\end{table}

\paragraph{Note:}
Modules M1, M2, M4, M7, M10, M11, and M12 are not directly unit-tested due to the following reasons:

\begin{itemize}
  \item \textbf{M1} (Hardware-Hiding Module): Tested implicitly through I/O in M6 and system test T1.
  \item \textbf{M2} (Configuration Module): No computational logic; parameters are exercised via other modules.
  \item \textbf{M4} (Control Module): Acts as the main orchestrator; validated indirectly via end-to-end system tests.
  \item \textbf{M7} (Model Module): Uses standard GNN backbones from existing implementations and is assumed correct.
  \item \textbf{M10, M11, M12} (Pytorch Module, Pytorch Geometric Module, GUI Module): External libraries (i.e., Pytorch, Pytroch Geometric, and Matplotlib).
\end{itemize}


				
\bibliographystyle{plainnat}

\bibliography{../../refs/References}
%\bibliography{refs/References}

\newpage

\section{Appendix}

\subsection{Usability Survey Questions?}
\label{sub:survey}
Please rate the following statements on a scale from 1 (Strongly Disagree) to 5 (Strongly Agree):

\begin{enumerate}
    \item The system was easy to learn and use.
    \item The interface is clear and intuitive.
    \item The system responded in a reasonable amount of time.
    \item I felt confident using the system without needing extra help.
    \item Error messages, if any, were clear and helpful.
    \item I am satisfied with my overall experience using the system.
\end{enumerate}

\vspace{1em}

\noindent \textbf{Optional Open-ended Questions:}
\begin{enumerate}
    \item What did you like most about the system?
    \item What improvements would you suggest?
\end{enumerate}


\end{document}
