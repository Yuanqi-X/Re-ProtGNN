\documentclass[12pt, titlepage]{article}

\usepackage{booktabs}
\usepackage{tabularx}
\usepackage{hyperref}
\hypersetup{
    colorlinks,
    citecolor=blue,
    filecolor=black,
    linkcolor=red,
    urlcolor=blue
}
\usepackage[round]{natbib}

\input{../Comments}
%% Common Parts

\newcommand{\progname}{ProgName} % PUT YOUR PROGRAM NAME HERE
\newcommand{\authname}{Yuanqi Xue} % AUTHOR NAMES                  

\usepackage{hyperref}
    \hypersetup{colorlinks=true, linkcolor=blue, citecolor=blue, filecolor=blue,
                urlcolor=blue, unicode=false}
    \urlstyle{same}
                                



\begin{document}

\title{System Verification and Validation Plan for Re-ProtGNN} 
\author{\authname}
\date{\today}
	
\maketitle

\pagenumbering{roman}

\section*{Revision History}

\begin{tabularx}{\textwidth}{p{3cm}p{2cm}X}
\toprule {\bf Date} & {\bf Version} & {\bf Notes}\\
\midrule
Feb 24 & 1.0 & Initial Draft\\
\bottomrule
\end{tabularx}

~\\


\newpage

\tableofcontents

\listoftables

%section: list of figures
%\listoffigures

\newpage

\section{Symbols, Abbreviations, and Acronyms}

\renewcommand{\arraystretch}{1.2}
\begin{tabular}{l l} 
  \toprule		
  \textbf{symbol} & \textbf{description}\\
  \midrule 
  T & Test\\
  A & Assumption\\
  R & Requirement\\
  SRS & Software Requirements Specification\\
  ProtGNN & Prototype-based Graph Neural Network\\
  Re-ProtGNN & Re-implementation of the ProtGNN model\\
  GNN & Graph Neural Network\\
  GIN & Graph Isomorphism Network\\
  \bottomrule
\end{tabular}\\

\newpage

\pagenumbering{arabic}

This document outlines the Verification and Validation (VnV) plan for Re-ProtGNN, a prototype-based interpretable Graph Neural Network. The objective of this plan is to ensure that the system meets the functional and non-functional requirements specified in the Software Requirements Specification (SRS). The document is structured as follows: Section 2 provides general information about Re-ProtGNN, including its objectives and scope. Section 3 details the verification strategies, covering SRS verification, design verification, and implementation verification. Section 4 describes the system tests, including functional and non-functional testing. Finally, Section 5 will cover additional test descriptions as needed.

\section{General Information}

\subsection{Summary}

The software under validation is Re-ProtGNN, a model designed to enhance the interpretability of Graph Neural Networks. It aims to classify graph-structured data while generating prototypes that explain its predictions. The system consists of two main components:

\begin{itemize}
    \item Training Phase: Learns meaningful representations of graphs while optimizing a loss function to improve both classification accuracy and prototype relevance.
    \item Inference Phase: Uses the trained model to classify unseen graphs and generate a set of representative prototypes that provide human-interpretable explanations.
\end{itemize}

Re-ProtGNN is implemented in Python, utilizing PyTorch Geometric for graph learning and Pytest for automated testing.


\subsection{Objectives}

The main goal of this VnV plan is to verify the correctness of the program, which is to ensure that the system correctly processes graph-structured inputs, trains models effectively, and generates prototype-based explanations.

%\begin{itemize}
    %\item Correctness: Ensure that the system correctly processes graph-structured inputs, trains models effectively, and generates meaningful prototype-based explanations.
    %\item Accuracy: Assess whether the model attains a minimum classification accuracy of 80\% on the MUTAG dataset and confirm that the selected prototypes align with expected graph representations.
    %item Interpretability: Validate that the extracted prototypes contribute to a clearer understanding of the model’s decision-making process.
%\end{itemize}

Out-of-Scope Objectives:
\begin{itemize}
    \item External Library Verification: Core dependencies such as PyTorch and Torch-Geometric are presumed to be reliable and are not explicitly tested in this plan.
    \item Graph Encoder Validation: Graph encoders, including Graph Isomorphism Networks (GINs), are adopted from prior research, and their correctness is assumed without additional validation.
\end{itemize}

\subsection{Challenge Level and Extras}
This is a research project, but extras may be included if time permits.

\subsection{Relevant Documentation}

The Re-ProtGNN project is supported by several key documents that ensure the system is properly designed, implemented, and validated. These documents include:

\begin{itemize}
    \item Problem Statement: This document~\cite{yuanqi2025protgnn} introduces the motivation of Re-ProtGNN and the core problem it aims to solve.
    
    \item Software Requirements Specification (SRS): The SRS~\cite{Yuanqi_ReProtGNN_SRS} defines the functional and non-functional requirements of Re-ProtGNN, and it also outlines the expected behavior of the system.
\end{itemize}

\section{Plan}

This section outlines the Verification and Validation (VnV) plan for Re-ProtGNN. It starts with an introduction to the verification and validation team (Subsection~\ref{sec:vvt}), detailing the members and their roles. Next, it covers the SRS verification plan (Subsection~\ref{sec:srsvp}), followed by the design verification plan (Subsection ~\ref{sec:dvp}). The document then presents the VnV verification plan (Subsection~\ref{sec:vvp}) and the implementation verification plan (Subsection~\ref{sec:ivp}). Finally, it includes automated testing and verification tools (Subsection~\ref{sec:att}) and concludes with the software validation plan (Subsection~\ref{sec:svp}).
